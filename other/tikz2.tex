%Преамбула
\documentclass[a4paper,12pt,oneside]{article}

\usepackage{geometry}
\geometry{left=3cm,right=2cm,top=3cm,bottom=3cm,bindingoffset=0cm}
\usepackage{python}
\usepackage{cmap}
\usepackage[T2A]{fontenc}
\usepackage[utf8x]{inputenc}
\usepackage[english, russian]{babel}
\usepackage{misccorr}
\usepackage{amssymb,amsfonts,amsmath,amsthm}  % математические дополнения от АМС
\usepackage{envmath}  % многострочные формулы EqSystem
\usepackage{indentfirst} % Включение отступа первой строки раздела
\usepackage[usenames,dvipsnames]{color} % названия цветов


\usetikzlibrary{%
    decorations.pathreplacing,%
    decorations.pathmorphing,%
    arrows,%
    patterns
}
\newcommand{\Scale}{1}
\newcommand{\lft}{9}
\newcommand{\rft}{10.43*1.5}
\newcommand{\Xstep}{1.5 }
\newcommand{\Ystep}{1.5 *20}
\newcommand{\Radius}{0.1}
\newcommand{\Color}{black}

%Содержимое документа
\begin{document}
	\begin{figure}[tb]
		% \centering	
		\hspace{-1.5cm}	
	\begin{tikzpicture}%[scale=\Scale] % легко изменить масштаб
	\draw[fill,yellow!60] 	
							(1.5,0) --
							(3.0000,1.0000) --
							(4.5000,2.0000) --
							(6.0000,3.0000) --
							(6.7500,4.0000) --
							(7.5000,5.0000) --
							(9.0000,6.0000) --
							(9.7500,7.0000) --
							(10.5000,8.0000) --
							(12,\lft) --
							(\rft,\lft) --
							(\rft,-\lft) --
							(15.0000,-8.0000) --
							(13.5000,-7.0000) --
							(10.5000,-6.0000) --
							(9.0000,-5.0000) --
							(6.7500,-4.0000) --
							(5.2500,-3.0000) --
							(3.7500,-2.0000) --
							(2.2500,-1.0000) -- cycle;
				

	\draw[step=.1cm , gray!20] (0,-\lft) grid (\rft ,\lft);
	\draw[step=0.5cm , gray!50] (0,-\lft) grid (\rft ,\lft);
	\draw[step=1cm , gray!80] (0,-\lft) grid (\rft ,\lft);

	\draw[-{>[scale=1.0]}] (-0,0) -- (10*1.5 +1 ,0) node[anchor=north west] {\scalebox{1.5}{$x$}};
	\draw[->] (0,-\lft) -- (0,\lft+1 ) node[anchor=east] {\scalebox{1.5}{$y$}};

	\foreach \x in {1,2,3,4,5,6,7,8,9,10} {
	\draw (\x*\Xstep,0.05) -- (\x*\Xstep,-0.05);
	\draw (\x*\Xstep,0) node[anchor=north] {\scalebox{1.5}{$\x$}};
	}

	\foreach \y in {-0.3,-0.2,-0.1, 0, 0.1,0.2,0.3} {
	\draw (0.05,\y*\Ystep) -- (-0.05,\y*\Ystep);
	\draw (0,\y*\Ystep) node[anchor=north west] {\scalebox{1.5}{$\y$}};
	}

	% \draw[blue] (15,0.5) node[anchor=east] {\scalebox{1.4}{Синхронизованная зона}};
	% \draw[blue] (9.5,8.2) node[anchor=east] {\scalebox{1.4}{Несинхронизованная зона}};
	% \draw[blue] (9.5,-8.3) node[anchor=east] {\scalebox{1.4}{Несинхронизованная зона}};
	\input{table.out}

	\end{tikzpicture}
	\caption{Зоны синхронизации}
	\label{fig:figure1}
\end{figure}
\end{document}