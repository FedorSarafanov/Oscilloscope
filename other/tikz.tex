%Преамбула
\documentclass{article}

\usepackage{geometry}
\geometry{left=2cm,right=2cm,top=3cm,bottom=3cm,bindingoffset=0cm}

\usepackage{cmap}
\usepackage[T2A]{fontenc}
\usepackage[utf8x]{inputenc}
\usepackage[english, russian]{babel}
\usepackage{misccorr}
\usepackage{amssymb,amsfonts,amsmath,amsthm}  % математические дополнения от АМС
\usepackage{envmath}  % многострочные формулы EqSystem
\usepackage{indentfirst} % Включение отступа первой строки раздела
\usepackage[usenames,dvipsnames]{color} % названия цветов

\usepackage{tikz}


\usetikzlibrary{%
    decorations.pathreplacing,%
    decorations.pathmorphing%
}
\newcommand{\Scale}{2.4}
\newcommand{\lft}{6/\Scale}



%Содержимое документа
\begin{document}

	\begin{figure}[tb]
		\centering
	\begin{tikzpicture}[scale=\Scale, every node/.style={scale=\Scale}] % легко изменить масштаб

	\draw[step=.1cm/\Scale, gray!20] (0,-\lft) grid (11/\Scale,\lft);
	\draw[step=0.5cm/\Scale, gray!50] (0,-\lft) grid (11/\Scale,\lft);
	\draw[step=1cm/\Scale, gray!80] (0,-\lft) grid (11/\Scale,\lft);


	\draw[->] (-0,0) -- (11/\Scale+1/\Scale,0) node[anchor=north west] {\scalebox{0.6}{$x$}};
	\draw[->] (0,-\lft) -- (0,\lft+1/\Scale) node[anchor=east] {\scalebox{0.6}{$y$}};

	\foreach \x in {1,2,3,4,5,6,7,8,9,10,11} {
	\draw (\x/\Scale,0.05) -- (\x/\Scale,-0.05);
	\draw (\x/\Scale,0) node[anchor=north] {\scalebox{0.6}{$\x$}};
	}

	\foreach \y in {-0.3,-0.2,-0.1, 0, 0.1,0.2,0.3} {
	\draw (0.05,\y/\Scale*20) -- (-0.05,\y/\Scale*20);
	\draw (0,\y/\Scale*20) node[anchor=east] {\scalebox{0.6}{$\y$}};
	}

	% \draw (0,0) node[anchor=north east] {\scalebox{0.6}{$0$}};
	% \draw[black] plot[] file{sin2.table};
	% \draw[black,thick,xscale=0.3] plot[mark=o] file{sin.table};
	% \draw[densely dashed,thin] (0,1.57) -- (3,1.57);
	% \draw (0,1.57) node[anchor=north east] {\scalebox{0.6}{$0.15$}};
	% \draw[densely dashed,thin] (0,-1.57) -- (3,-1.57);
	% \draw (0,-1.57) node[anchor=north east] {$-\frac{\pi}{2}$};
	\input{sin.table}
	\end{tikzpicture}
	\caption{Зоны синхронизации}
	\label{fig:figure1}
\end{figure}
% \hspace{-2.5cm}
\end{document}



\documentclass{article}


\usepackage{tikz}
\usetikzlibrary{%
    decorations.pathreplacing,%
    decorations.pathmorphing%
}
\begin{document}
\pagestyle{empty}

\begin{tikzpicture}[
    media/.style={font={\footnotesize\sffamily}},
    wave/.style={
        decorate,decoration={snake,post length=1.4mm,amplitude=2mm,
        segment length=2mm},thick},
    interface/.style={
        % The border decoration is a path replacing decorator. 
        % For the interface style we want to draw the original path.
        % The postaction option is therefore used to ensure that the
        % border decoration is drawn *after* the original path.
        postaction={draw,decorate,decoration={border,angle=-45,
                    amplitude=0.3cm,segment length=2mm}}},  
    interface2/.style={
        % The border decoration is a path replacing decorator. 
        % For the interface style we want to draw the original path.
        % The postaction option is therefore used to ensure that the
        % border decoration is drawn *after* the original path.
        postaction={draw,decorate,decoration={border,angle=45,
                    amplitude=0.3cm,segment length=2mm}}},                                        
    ]
    % Round rectangle
    % \fill[gray!10,rounded corners] (-4,-3) rectangle (4,0);
    % Interface
    \draw[step=.1cm, gray!30] (0,0) grid (10,10);
    % \draw[blue,line width=.5pt,interface](0,0)--(-4,0);
    % \draw[blue,line width=.5pt,interface2](0,0)--(0,1);
     % Vertical dashed line
    % \draw[dashed,gray](0,-5)--(0,0);
    % Coordinates system
    % \draw(0,0.15)node[above]{$x$};
    % \draw[<->,line width=.5pt] (6,0) node[above]{$y$}-|(0,-6) node[left]{$x$};
% 

    % \draw[->] (0,-1.5)arc(-90:45:1.5cm);
    % Media names
    % \path[media] (-3,.6)  node {media 1}
    %              (-3,-.6) node {media 2};

    % $x$ axis
    % \filldraw[fill=white,line width=1pt](0,0)circle(.12cm);
    % \draw[line width=.6pt] (0,0)
    %                       +(-135:.12cm) -- +(45:.12cm)
    %                       +(-45:.12cm) -- +(135:.12cm);
    % Interface pointer
    % \draw[-latex,thick](3.2,0.5)node[right]{$\mathsf{S_{1,2}}$}
    %      to[out=180,in=90] (3,0);

    % \draw[dashed, very thin] (-120:5)arc(-120:0:5);
    % \draw[<->, thin] (-120:2) arc(-120:-90:2);
    % \path(-105:2.5) node[above]{$\alpha$};

    % \draw[line width=1pt] (0,0)--(-120:5);
    %  	\filldraw[fill=black!10,line width=1pt](-120:5)circle(.12cm);

    % \draw[line width=1pt] (0,0)--(0:5);
    % 	 \filldraw[fill=black!10,line width=1pt](5,0)circle(.12cm);    

    % To-paths are really useful for drawing curved lines. The above
    % to path is equal to:
    %
    % \draw[-latex,thick](3.2,0.5)node[right]{$\mathsf{S_{1,2}}$}
    %      ..controls +(180:.2cm) and +(up:0.25cm) .. (3,0);
    % Internally the to path is translated to a similar bezier curve,
    % but the to path syntax hides the complexity from the user. 
\end{tikzpicture}


\end{document}